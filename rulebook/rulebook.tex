\documentclass[twoside]{minimal}

\usepackage[letterpaper, margin=0.75in]{geometry}
\usepackage[ampersand]{easylist}
\usepackage{graphicx}
\input{graphicspath.tex}

\usepackage{wrapfig}
\usepackage[usexcolor, usetransparent]{svg}
\svgpath{svg/}

\usepackage{caption}
\usepackage{float}

% Diagram floats
\floatname{diagram}{}
\newfloat{diagram}{tbph}{diag}

\renewcommand{\familydefault}{\sfdefault}
\renewcommand{\normalsize}{\fontsize{18}{22}\selectfont}

% Example usage:
% \begin{section}{section title} \title
\newenvironment{section}[1]
{\renewcommand{\title}{\begin{center} \fontsize{24.88}{32}\selectfont #1 \end{center}} \ignorespaces}
{\clearpage \ignorespacesafterend}

% Example usage:
% \begin{supersection}{supersection title} \begin{section}{subsection} \title
\newenvironment{supersection}[1]
{
\renewcommand{\title}{\begin{center} \fontsize{24.88}{32}\selectfont #1 \end{center}}
\renewenvironment{section}[1]
 {\title \renewcommand{\title}{\begin{flushleft} \fontsize{20.74}{28}\selectfont ##1 \end{flushleft}} \ignorespaces}
 {\clearpage \ignorespacesafterend}
\ignorespaces
}
{\clearpage \ignorespacesafterend}

\newcommand{\side}[4]{
\begin{wrapfigure}{#1}{0.35\textwidth}
  \begin{center}
    #2[width=0.25\textwidth]{#3}

    \caption*{#4}
  \end{center}
\end{wrapfigure}
}

\newcommand{\sideimage}[3]{\side{#1}{\includegraphics}{#2}{#3}}
\newcommand{\sidesvg}[3]{\side{#1}{\includesvg}{#2}{#3}}

\newcommand{\full}[3]{
\begin{diagram}
    \begin{center}
    {#2[width=\textwidth, height=\textheight]{#1}}
    \caption*{#3}
    \end{center}
\end{diagram}
}
\newcommand{\fullimage}[2]{\full{#1}{\includegraphics}{#2}}
\newcommand{\fullsvg}[2]{\full{#1}{\includesvg}{#2}}

\begin{document}

\begin{section}{Retrograde}
\title

Welcome to Retrograde, the game of interplanetary domination.
This guide will show you all you need to know to learn how to play the game.

\end{section}

\begin{section}{Before You Begin}
\title

\sideimage{r}{Earth00.png}{A Homeworld - Earth}

Construct your deck by selecting a \emph{Homeworld} and 30 other cards.
Order them in your deck such that you draw what you want when you think you’ll want it.
There is no shuffling before the game begins.

You and your opponent place your Homeworld face-up on the field, in the center of the asset line.
These represent your currently controlled assets.
New assets will be added during the game to the left and right of these first assets.
Draw a starting hand of 5 cards.
Flip a coin to determine who goes first.
Place your orbit tracker at the top of your respective Homeworlds.

\fullsvg{starting-board-state.svg}{An example of a starting board state.}

\end{section}

\begin{section}{Turn Order}
\title
    \begin{easylist}[enumerate]
        & Upkeep

            && The orbit counter advances by 1.
            && Each of your spacecraft heals any damage it sustained last turn.
            && Each of your spacecraft regenerates any delta-vee it used last turn.

        & Maneuvers

            Any card capable of making maneuvers may do so now.

        & Research

            Any card capable of researching may do so now.

        & Repair

            Any card capable of repairing may do so now.

        & Construction

            Any card capable of repairing may do so now.

        & End of Turn

            Any end of turn effects trigger here.
            All of your tapped cards become untapped.
            If your opponent has no Homeworld, then you win the game.

    \end{easylist}
\end{section}

\begin{supersection}{Upkeep}
    \begin{section}{Increment Orbit Counter}
        \title

        The orbit counter moves one space clockwise around the border of your Homeworld.

        \fullsvg{increment-orbit-counter.svg}{Incrementing the orbit counter}
    \end{section}

    \begin{section}{Heal Spacecraft Damage}
        \title

        Remove any damage counters from your spacecraft.

        \fullsvg{heal-spacecraft-damage.svg}{Healing spacecraft damage}
    \end{section}

    \begin{section}{Regenerate Spacecraft Delta-Vee}
        \title

        Remove any delta-vee counters from your spacecraft.

        \fullsvg{regenerate-delta-vee.svg}{Regenerating spacecraft delta-vee}
    \end{section}
\end{supersection}

\begin{supersection}{Maneuvers}
    \begin{section}{Orbital Burn}
        \title

        An orbital burn moves a spacecraft to an orbital directly above or below the orbital it is currently in.
        It costs an amount of delta-vee equal to the number on the target orbital.

        In the below diagram the spacecraft L.A.V. spends 4 delta-vee to move from the second orbital on the Supergiant asset to the first.
        This is the cost listed on the asset card for moving into the first orbital.

        \fullsvg{orbital-burn.svg}{In the diagram the spacecraft L.A.V. spends 4 delta-vee to move from the second orbital on the Supergiant asset to the first. This is the cost listed on the asset card for moving into the first orbital.}
    \end{section}

    \begin{section}{Transfer Burn}
        \title

        A transfer burn moves a spacecraft from the outermost orbital on the asset it is currently orbiting to the outermost orbital on another asset.
        The transfer burn may only be initiated when the spacecraft is in the outermost orbital of an asset.
        The transfer burn costs an amount of delta-vee equal to: the currently active orbit track value on the destination asset minus the currently active orbit track value on the starting asset.
        A transfer burn always costs at least 3 delta-vee.

        \fullsvg{transfer-burn.svg}{A Fast Attack Craft performs a transfer burn.}
    \end{section}

    \begin{section}{Scuttle}
        \title

        The scuttle maneuver costs 1 delta-vee and immediately destroys the spacecraft.
        It becomes debris in the orbital it was in before it was destroyed.

        \fullsvg{scuttle.svg}{
            The Light Bomber currently has 1 delta-vee left.
            It spends it to scuttle in the first orbital of Research Station 2321, creating a dangerous debris field for any other ships that will occupy that orbital.
        }
    \end{section}

    \begin{section}{Combat}
        \title

        Combat is initiated between two spacecraft in orbit around the same asset.
        To begin combat, spend an amount of delta-vee that you wish to commit on the combat maneuver.
        The more delta-vee you spend, the higher your chances of making a hit during the combat.
        Additionally, attackers that are above their target add 1 bonus delta-vee to their attack for each orbital between them and their target.

        \fullsvg{combat-attacker-above.svg}{Combat with the attacker above its target.}
    \end{section}

    \begin{section}{Combat (cont.)}
        \title

        Shooting is best done from above your opponent.
        If you're attacking from below you will need to spend extra delta-vee to hit your target.
        Attackers from below must spend one extra delta-vee to attack for each orbital between them and their target.

        \fullsvg{combat-attacker-below.svg}{Combat with the attacker below its target.}
    \end{section}

    \begin{section}{Bombardment}
        \title

        A spacecraft may initiate a bombardment maneuver by spending an amount of delta-vee.
        Bombardment targets the asset currently being orbited or an upgrade on the asset.
        When an asset has damage on it equal to or greater than its bombardment resistance, it is considered bombarded.
        Bombarded assets immediately swap control to the player who bombarded the asset.
        Any upgrades that were attached to the asset are destroyed.
        Any in progress construction on the asset is destroyed and removed from the game.

        \fullsvg{bombardment.svg}{Asset bombardment.}

    \end{section}
\end{supersection}

\begin{section}{Destroyed Cards}
\title

Destroyed spacecraft are flipped face down and become debris fields

\fullsvg{destroyed-spacecraft.svg}{Flipping a spacecraft when it is destroyed}

Destroyed upgrades are removed from the game.
Cards that are removed from the game are placed face down out of the way of the playing field.
They may not be looked at by any player other than their owner.
\end{section}

\begin{section}{Debris}
\title

When a spacecraft is destroyed it becomes debris.
At the end of your turn, each of your spacecraft with zero delta-vee in an orbital that contains debris takes 1 damage for each debris in an orbital.

\fullsvg{debris.svg}{A spacecraft taking damage from debris}
\end{section}

\begin{section}{Research}
\title

After maneuvers have been finished any untapped asset of yours may be tapped to perform research.
When you perform research with an asset, draw 1 card from your deck.
\end{section}

\begin{section}{Repair}
\title

After research has been finished any untapped asset of yours may be tapped to perform repairs.
When you perform repairs with an asset remove 1 damage from the asset or any upgrade on the asset.

\fullsvg{repair.svg}{
    Earth has sustained 2 damage.
    It performs repairs during the turn to remove 1 damage, leaving it with a total of 1 damage on it.
}
\end{section}

\begin{section}{Construction}
\title

After repair has been finished any untapped asset of yours may be tapped to perform construction.
When an asset performs construction it allocates its full construction power to one of three things:

\begin{easylist}[itemize]
    & A spacecraft under construction on the asset.
    & An upgrade under construction on the asset.
    & A friendly asset currently being constructed on the field.
\end{easylist}

\end{section}

\begin{section}{Constructing Spacecraft}
\title

Spacecraft are constructed by placing them face down next to the asset.
When construction power is allocated to it, place a number of construction counters on it equal to the construction power of the asset.
During the construction phase, if a spacecraft has a number of construction counters on it equal to or greater than its construction cost then you may complete construction and it enters play face up in the lowest orbital of the asset.
You do not have to complete construction of a spacecraft as soon as it is available.
The asset must have a facility that matches the spacecraft class in order to allocate construction to it.
Only one spacecraft may be under construction at a time.

If a spacecraft could complete construction, but the combined population of all of your spacecraft currently in play and the new spacecraft would be greater than the combined population of all of your assets in play, then the spacecraft may not complete construction.

\fullsvg{constructing-spacecraft.svg}{A Missile Frigate under construction}
\end{section}

\begin{section}{Construction Upgrades}
\title

Upgrades are constructed by placing them face down under the asset.
When construction power is allocated to it, place a number of construction counters on it equal to the construction power of the asset.
During the construction phase, if an upgrade has a number of construction counters on it equal to or greater than its construction cost then you may complete construction and it enters play face up under the asset.
You do not have to complete construction of an upgrade as soon as it is available.
Multiple upgrades may be in play on an asset at the same time, though only one may be under construction at a time.

\fullsvg{constructing-upgrades.svg}{Booster Rockets under construction}
\end{section}

\begin{section}{Constructing Assets}
\title

Assets are constructed by placing them face down in the assets play area.
Any constructed asset may allocate its construction power to the unconstructed asset.
When construction power is allocated to it, place a number of construction counters on it equal to the construction power of the asset that allocated the construction power.
During the construction phase, if an asset has construction counters on it equal to or greater than its construction cost it may complete construction and is flipped face up and enters play.
You do not have to complete construction of an asset as soon as it is available.
Multiple assets may be under construction at the same time.

If an enemy spacecraft is in orbit around an asset, the asset is considered blockaded and it may not allocate its construction to another asset.

Assets that are under construction are not in play and may not be orbited.

\fullsvg{constructing-assets.svg}{An asset under construction}

\end{section}

\begin{section}{Asset Anatomy}
\title

\fullsvg{asset-anatomy.svg}{Anatomy of Jupiter}

\end{section}

\begin{section}{Upgrade Anatomy}
\title
\fullsvg{upgrade-anatomy.svg}{Anatomy of Fortified Buildings}
\end{section}

\begin{section}{Spacecraft Anatomy}
\title
\fullsvg{spacecraft-anatomy.svg}{Anatomy of Light Bomber}
\end{section}

\begin{section}{Play Area}
\title
\fullsvg{play-area.svg}{}
\end{section}
\end{document}
